\documentclass[a4paper]{ctexart}
\usepackage[top=2.1cm,bottom=2cm,left=1.9cm,right=1.9cm]{geometry} 
\usepackage{amsmath} 
\usepackage{booktabs}
\usepackage{amsthm}
\usepackage{longtable} 
\usepackage{graphicx}
\usepackage{subfigure}
\usepackage{caption}
\usepackage{fontspec}
\usepackage{titlesec}
\usepackage{fancyhdr}
\usepackage{subfig}
\def\degree{$^{\circ}$}
\def\mm{\mathrm{mm}}
\def\cm{\mathrm{cm}}
\def\nm{\mathrm{nm}}
\def\V{\mathrm{V}}
\def\m{\mathrm{m}}
\def\g{\mathrm{g}}
\def\s{\mathrm{s}}
\def\i{\mathrm{i}}
\def\A{\mathrm{A}}
\def\mA{\mathrm{mA}}
\def\mV{\mathrm{mV}}
\def\mT{\mathrm{mT}}
\title{\textbf{观察光的偏振现象}}
\author{王崇斌 1800011716}
\date{}
\makeatletter %使\section中的内容左对齐
\renewcommand{\section}{\@startsection{section}{1}{0mm}
	{-\baselineskip}{0.5\baselineskip}{\bf\leftline}}
\makeatother
\begin{document}
	\pagestyle{fancy}
	\lhead{普通物理实验报告} 
	\chead{}
	\rhead{}
	\maketitle
    \thispagestyle{fancy}
    \section{\large{实验现象的描述与解释}}
    \subsection{用偏振光镜验证布儒斯特定律}
    \par 
    (实验装置参考课本)实验中首先要调节激光的入射角为布儒斯特角,由于事先知道此种玻璃的
    $i_{B} = 57^\circ$,那么调节激光束与水平面夹角为$24^\circ$,反射镜P与竖直方向夹角为
    $33^\circ$。这样就基本保证了反射光束是线偏振光(根据菲涅尔反射定律,只有s波),将玻片
    堆A调节至与反光镜平行。
    \paragraph{实验现象}
    在水平面上转动(绕$z$轴)玻片堆A,可以看到反射光的强度呈周期性变化:每旋转一周可以观察到两次光、
    强极小与极大的情况,光强极大值基本出现在AP平行时,光强极小出现在A相对平行位置旋转
    $90^\circ$时。同时观察到透射光强的变化趋势与之恰好相反,这也是能量守恒的结果(忽
    略介质对光的吸收或者认为介质对光的吸收是各向同性的,与光矢量的方向无关)。绕$z$轴
    转动直至反射光消光(即前面所述的光强极小值的情况),将玻片堆绕水平方向转轴$y$旋转,
    可以观察到无论是向那个方向,只要是偏离消光位置,反射光的光强均会增大,相应的透射光
    的光强均会减小。将一偏振片放在反光镜与玻片堆中间,转动偏振片,观察到在某个特定角度
    下会出现消光现象,这说明了此时反光镜产生的反射光为一束线偏振光。
    \paragraph{现象解释}
    由菲涅尔折射反射公式可知:
    $$
    \widetilde{r}_{p} = \frac{n_{2}\cos{i_{1} - n_{1}\cos{i_{2}}}}{n_{2}\cos{i_{1}} + n_{1}\cos{i_{2}}}
    $$
    那么使得$\widetilde{r}_{p}=0$的入射角会使反射光只有$s$分量(即与介质表面平行的
    分量),即为一线偏振光,我们把这样的入射角称作布儒斯特角,记作$i_{B}$。
    \par 
    在保证AP平行后将玻片堆绕$z$转动,这样保证了入射向玻片堆的光入射角为布儒斯特角,那么
    反射过程中只有$s$分量被反射,由于入射光为一线偏振光,很容易看出当玻片堆从平行位置
    转过$90^\circ$后,入射光只有$p$分量,那么反射光消光。但是事实上并不能观察到反射光
    完全消光的现象,即便在微调玻片堆的角度之后也不行,个人猜测有可能与玻片堆表面不平整
    或沾有其他折射率有差别的污渍而导致的。
    \par 
    至于透射光,情况较为复杂,通过菲涅尔公式很容易计算出在入射角为$i_{B}$的情况下$s$,$p$
    波的透射率为:
    $$
    \widetilde{t}_{p} = \frac{2n_{1}\cos{i_{1}}}{n_{2}\cos{i_{1}} + n_{1}\cos{i_{2}}} = \frac{n_{1}}{n_{2}}
    $$
    $$
    \widetilde{t}_{s} = \frac{2n_{1}\cos{i_{1}}}{n_{1}\cos{i_{1}} + n_{2}\cos{i_{2}}} = \frac{2n_{1}^{2}}{n_{1}^{2} + n_{2}^{2}}
    $$
    可见透射光并不是一束线偏振光,而是部分偏振光(由于两个正交分量相位之间无关联,所以
    不能生成椭圆偏振光,只能是部分偏振光),同时还能看出$\widetilde{t}_{p}\ge \widetilde{t}_{s}$。
    此时假设玻片堆由$N$个相同玻片组成,那么容易得到经过$N$个玻片后的$s,\;p$波的透射率为:
    $$
    \widetilde{t}_{p}(N) = \left(\frac{n_{1}}{n_{2}}\right)^{N}
    $$
    $$
    \widetilde{t}_{s}(N) = \left(\frac{2n_{1}^{2}}{n_{1}^{2} + n_{2}^{2}}\right)^{N}
    $$
    那么我们可以得到正交分量的光强比值:
    $$
    \lambda = \frac{\widetilde{t}_{p}(N)}{\widetilde{t}_{p}(N)} = \left(\frac{n_{1}^{2} + n_{2}^{2}}{2n_{1}n_{2}}\right)^{N}
    $$
    容易看出在$n_{1}\ne n_{2}$时,随着$N$增大,透射光中$p$波所占比例逐渐增大,当玻片
    足够多时,可以认为透射光是一束线偏振光。
    \par 
    当玻片堆绕$y$轴旋转后,由于入射角偏离$i_{B}$,那么$p$波反射率不在为0,反射光增强
    与透射光减弱就是很自然的了。
    \par 
    上面的实验现象均与理论分析一致,说名此实验可以验证布儒斯特定律。
    \subsection{观察双折射现象}
    \par 
    方解石1:通过观察发现这个方解石的表面(透光面)应该是由自然解理形成的,呈现出一个
    非常规则的具有三重对称性的平行六面体,那么这个晶体的透光面应该不与光轴垂直(三重轴
    为光轴)。方解石2:与方解石1几乎一样,只不过在垂直于三重轴的位置磨出两个透光平面,
    仔细观察这两个平面没有完全磨光的地方,那里非常粗糙,而且呈现出小的突刺结构,这一定
    不是方解石自然解理的方向。这两个透光面垂直于光轴。
    \par 
    将方解石1放置在小孔上,可以观察到小孔通过晶体呈现出两个大小相同的像,旋转
    晶体,可以看到其中一个像固定不动,另外一个像随方解石同步旋转。在方解石这类
    各向异性的介质中,平行光轴方向与垂直光轴平面内的介电常数不同,通常将晶体中传播的光
    这样分类:将光矢量分解为与光轴和传播方向所在平面垂直的光,称为寻常光,寻常光的折射率
    与传播方向无关,因此在方解石1的两个平行表面之间成的像不随方解石旋转而旋转;将光矢量
    分解为在光轴与传播方向平面内的光矢量,这束光的折射率与传播方向与光轴的夹角有关,
    因此在旋转晶体(旋转光轴)时所成的像会随着光轴而旋转。
    \par 
    将方解石2与光轴垂直的平面放在小孔上,此时只能看到一个小孔的像。这是因为光是沿着光轴
    传播,因此光矢量只分布在垂直于光轴的平面内,只有寻常光分量,不会发生双折射。
    \par 
    将方解石1放在小孔上,上面再加上偏振片,旋转偏振片可以观察到两个像依次消光,其间偏振片
    方向转过$90^\circ$,证明了两个像是由线偏振光所成,且两个线偏振光的偏振方向正交,这
    与寻常光、非寻常光光矢量正交是一致的。
    \subsection{观察线偏振光通过$\frac{\lambda}{2}$波晶片后的现象}
    \par 
    在眼睛与钠光灯之间放置起偏器P,旋转起偏器,观察透射光的强度,发现旋转一周光强没有
    眼睛可观测的变化,说明光源发出的光是一束自然光(圆偏振光不太可能,就实验时的钠光灯
    而言)。在起偏器后方放置一检偏器,旋转检偏器一周可以观察到两次消光。在两个偏振片
    之间插入$\frac{\lambda}{2}$波晶片,将其旋转一周,可以观察到4次消光。由于$\frac{\lambda}{2}$
    波晶片可以将线偏振光变为相对光轴对称的线偏振光,因此旋转波晶片时相当于以两倍的角速度
    旋转透过波晶片的线偏振光,这样旋转波晶片一周,透过波晶片的线偏振光旋转两周,产生4
    次消光。
    \par 
    调节起偏器与检偏器正交,此时插入$\frac{\lambda}{2}$波晶片,旋转使其消光,此时波晶片
    的光轴与某一个偏振片的透振方向相同。记此时起偏器、检偏器的位置为零点(不改变波晶片位置
    ),向某一方向旋转起偏器,记转角为$\theta$,向另一个方向旋转检偏器直至第一次消光,
    记转角为$\theta^{'}$,下面给出实验中记录的数据表。
    \begin{table}[htbp]
        \centering
        \caption{$\frac{\lambda}{2}$波晶片转动影响偏振态的观察记录}
        \begin{tabular}{ccc}
            \toprule[1.5pt]
            $\theta$ & $\theta^{'}$ & 线偏振光经$\frac{\lambda}{2}$片后振动方向转过的角度 \\
            \midrule
            0\degree & 0\degree & 0\degree \\
            15\degree & -15\degree & 30\degree \\
            30\degree & -30\degree & 60\degree \\
            45\degree & -45\degree & 90\degree \\
            60\degree & -60\degree & 120\degree \\
            75\degree & -75\degree & 150\degree \\
            90\degree & -90\degree & 180\degree \\
            \bottomrule[1.5pt]
        \end{tabular}
    \end{table}
    \par 
    从实验记录表中可以看出,$\frac{\lambda}{2}$波晶片确实使得通过其的线偏振光依然为
    线偏振光,并且从数据中可以看出此波晶片可以使得线偏振光变为相对于其光轴对称的线偏振光
    。
    \subsection{检验椭圆偏振光与部分偏振光}
    \paragraph{椭圆偏振光与部分偏振光的产生}
    由前面的讨论可以知道,让自然光通过玻片堆就可以产生一束部分偏振光,部分偏振光通过偏振片
    后不会消光,会出现光强极大与极小的两个方向,这一点与椭圆偏振光通过偏振片完全一致,因此
    不能通过一个偏振片区分这两种偏振光。对于椭圆偏振光,让自然光通过偏振片,产生一束线偏振
    光,让此线偏振光通过一$\frac{\lambda}{4}$,只要不是运气足够好,都可以产生一束椭圆
    偏振光。
    \paragraph{检验椭圆偏振光与部分偏振光}
    假设已经产生了这两种光,下面以椭圆偏振光为例进行讨论。在光路中放置一偏振片,旋转
    偏振片观察现象,如果光强没有明显变化,说明这是一束圆偏振光,只要稍微调节波晶片的角度
    即可,如果产生消光也调节波晶片的方向。如果观察到透射光的光强有明显的大小变化并且
    不会消光时,旋转找到透射光最弱的位置,那么此时第一个偏振片的透振方向与椭偏光短轴平行,
    此时放上第二个偏振片并旋转消光,那么这两个偏振片的透振方向就是椭圆的长短轴方向,此时
    在两个偏振片之间加上$\frac{\lambda}{4}$波晶片,旋转使其再次消光,那么波晶片光轴的
    方向与某一个偏振片透振方向平行,则这个波晶片相对于椭偏光是正方的,去掉第一个偏振片
    直接通过波晶片的椭偏光就变为线偏振光,此时旋转第二个偏振片,可以看到消光现象。
    \par
    如果对部分偏振光重复上面的操作,由于无论$\frac{\lambda}{4}$波晶片如何摆放,透过光束
    一定是部分偏振光,即旋转第二个偏振片不能观察到消光现象,这样就可以区分部分偏振光
    与椭圆偏振光。\\
    \\
    \section{\large{分析与讨论}}
    \par 
    在检验椭圆偏振光时,首先要通过第一个偏振片观察部分消光现象,此时为了更好地确认椭圆的
    方位,应该选择观察光强极小值的位置。这是由人眼的特点决定的,因为人眼会对光强变化做出
    响应,对光线暗时的光强变化更加敏感。\\
    \\
    \section{\large{收获与总结}}
    \par 
    至此,一学期的普物实验结束了,这其中有苦有乐,也收获了不少东西。有时会因为实验报告
    复杂而占用时间而对普物实验心怀怨气,但也会在看到自己精彩的实验报告之后洋洋得意,
    心情舒畅。总之,很感谢老师们细心认真地教我实验,也感谢同学们的陪伴,多年之后,看到
    最后留下的那张照片,一定是一种很幸福的回忆。


\end{document}